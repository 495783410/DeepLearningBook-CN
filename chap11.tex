%%%%%%%%%%%%%%%%%%%%%%%%%%%%%%%%%%%%%%%%%%%%%%%%%%%%%%%%%
%%%%%%%%%%%%%%%%%%%author:rickymf4%%%%%%%%%%%%%%%%%%%%%%%
%%%%%%%%%%%%%%%%%%%%%%%%%%%%%%%%%%%%%%%%%%%%%%%%%%%%%%%%%
\chapter{实战方法}
\label{chap:11}

成功地应用深度学习技术远远不仅仅需要知道有哪些算法和这些算法的原理。一个好的机器学习实践者亦需要知道如何针对特定的应用选择算法,如何监控实验并对获得的反馈做出回应,从而优化该机器学习系统。在机器学习系统的日常开发中,实践者们需要决定是否需要搜集更多的数据,是增加还是减少模型的容量,是加上还是去掉规则化特征,是改进模型的优化,还是改进模型中的近似推理,亦或是对模型的实现程序进行调试。所有这些操作都是非常耗时的尝试,所以能够确定正确的行动方针是很重要的,而不是盲目猜测它。

这本书的大部分内容是关于各种机器学习模型、训练算法,和目标函数。这可能会造成这样的印象:成为机器学习专家最重要的本领是要了解各种机器学习技术并擅长不同类型的数学。 然而在实践中,正确地使用一个常见的算法比草率地使用一个晦涩的算法要好。算法的正确应用取决于掌握一些相当简单的方法。本章节中许多建议改编自\cite{Ng ( 2015 )}.

我们建议如下的实际设计过程:
\begin{itemize}
\item 确定您的目标 - 要使用的误度量和相应的目标价值。这些目标和误差度量应该由应用要解决的问题来驱动。
\item 尽快建立一个可以工作的端到端的流水线,包括适当估计它的性能指标。
\item 较好地给系统装上仪表来确定其性能瓶颈。诊断哪些组件的性能低于预期,以及是否是由于过度拟合、欠拟合,或数据或程序中存在着缺陷。
\item 基于来自仪表中特定的发现,重复地进行增量变更,如收集新数据,调整超参或更改算法。
\end{itemize}

我们将采用街景地址号码翻译系统作为一个运行的例子 (\cite{ Goodfellow et al. , 2014d })。 这个应用的目的是将建筑添加到谷歌地图中。街景车对建筑进行拍摄并记录照片对应的GPS坐标。通过卷积网络来识别每张照片中的地址号码,使得谷歌地图数据库可以将该地址加入到正确的位置上。关于如何开发这个商业应用程序的故事给出了如何遵循我们倡导的设计方法的例子。我们现在来描述这个过程中的每个步骤。

\section{性能指标}

确定目标(根据将采用的错误率指标)是必要的第一步,因为你的错误率指标将指导你以后的所有操作。你还应该知道你想要什么级别的性能。

请记住,对于大多数应用,不可能实现绝对零错误率。即使你有无限训练数据,并且可以恢复真实的概率分布,而贝叶斯误差定义了你可以希望实现的最小错误率。这是因为你的输入特征可能不包含有关输出变量的完整信息,或者因为系统可能是内在随机的。你也将受限于有限数量的训练数据。

由于各种原因,训练数据的量被限制了。当你的目标是构建一个一流的实际的产品或服务时,您通常可以收集更多数据,但必须确定进一步减少错误率的价值,并将其与收集更多数据的成本进行权衡。数据收集可能需要时间、金钱或使人遭受痛苦(例如,如果您的数据收集过程涉及侵入性的医学测试)。 而当你的目标是回答在固定的基准上哪个算法的表现更好时,通常是不允许收集更多数据作为基准的训练集的。

如何给性能水平确定一个合理的期望?通常,在做学术时,我们基于以前发布的基准测试结果,可以做一些错误率的估计。而在做实际项目时,我们考虑的错误率是要让我们的应用是安全的,成本效益的,或是吸引消费者的。一旦确定了实际期望的错误率,你的设计决策将以达到此错误率为指导。

