\chapter{处理配分函数}
\label{chap:18}
% Partition Function 配分函数的定义见 wiki:https://www.wikiwand.com/zh/%E9%85%8D%E5%88%86%E5%87%BD%E6%95%B0

%%%%%%%%%%%%%%%%%%%%%%%%%%%%%%%%%%%%%%%%%%%%%%%%%%%%%%%%%
%%%%%%%%%%%%%%%%%%% author:quxiaofeng %%%%%%%%%%%%%%%%%%%
%%%%%%%%%%%%%%%%%%%%%%%%%%%%%%%%%%%%%%%%%%%%%%%%%%%%%%%%%

如 \remove{16.2.2 节}中所见,\add{很多概率模型(一般称为无向图模型)}是用\consider{(非正态/未标准化/unnormalized)概率分布 \mbox{ \(\widetilde{p}(\bm{x};\bm{\theta})\)} }定义的。用配分函数\footnote{译者注:配分函数(Partition Function)的定义见 wiki:\url{https://www.wikiwand.com/zh/\%E9\%85\%8D\%E5\%88\%86\%E5\%87\%BD\%E6\%95\%B0}}\(Z(\bm{\theta})\)去除\(\widetilde{p}\)来得到一个概率分布:
\[p(\bm{x};\bm{\theta})=\frac{1}{Z(\bm{\theta})}\widetilde{p}(\bm{x};\bm{\theta}).\]