%%%%%%%%%%%%%%%%%%%%%%%%%%%%%%%%%%%%%%%%%%%%%%%%%%%%%%%%%
%%%%%%%%%%%%%%%%%%% author:KaiserW %%%%%%%%%%%%%%%%%%%%%%
%%%%%%%%%%%%%%%%%%% part:5.7-5.11  %%%%%%%%%%%%%%%%%%%%%%
%%%%%%%%%%%%%%%%%%%%%%%%%%%%%%%%%%%%%%%%%%%%%%%%%%%%%%%%%

\section{监督学习算法}
\label{sec:5.7}
前承\ref{sec:5.1.3}节,有监督学习(Supervised Learning)简单来讲就是一种学习算法,它会学着在某些输入和某些输出之间建立关联,这些输入\textbf{x}和输出\textbf{y}来自于训练集中的样本。很多时候,输出\textbf{y}很难自动采集,而必须由一位人工“监督者”(supervisor)提供,当然即便训练集的拟合目标已经自动采集完成,“监督学习”的名称仍然适用。
\subsection{概率监督学习}
\label{sec.5.7.1}
本书提到的大多数监督学习算法都是基于对概率分布$p(y|x)$的预测。我们可以简单地应用最大似然估计(maximum likelihood estimation)来找到分布$p(y|x;\theta)$参数族的最佳参数向量$\theta$。

已知线性回归(linear regression)对应参数族
\begin{equation}
	p(y|x;\theta) = \mathbb{N} (y;\theta^{T}, \textbf{\textit{I}})
  	\label{form:5.80}
\end{equation}

通过定义不同族的概率分布,我们可以将线性回归推广到分类(classification)情景。如果我们有两个类别,类0和类1,那么接下来只需确定其中一个类的的概率就可以了。类1的概率自然也就决定了类0的概率,因为两个概率值相加必然为1.
基于平均值将实数域上的正态分布进行参数化,这一分布我们也用于线性回归,这里的平均值可以是任意值。但是二元变量的分布则更加复杂一些,因为其平均值必然始终落在0和1之间。一种解决方案是应用逻辑函数(logistic function, 也称sigmoid函数)将线性函数的输出值挤压到(0,1)区间,转换后的值可以理解为是一个概率:
\begin{equation}
	p(y=1|x;\theta) = \sigma (\theta^{T}x)
  	\label{form:5.80}
\end{equation}

这一方法即是逻辑回归(logistic regression),这名字有些古怪,因为我们实际上用这个模型做分类而不是回归。
对于线性回归,我们可以解正规方程(normal equations)以求得最优权重。而逻辑回归就要复杂一些,它的最优权重没有解析解。我们只能通过最大化对数似然率(log-likelihood)来逼近最优解,具体的策略是,应用梯度下降法(gradient descent)使负对数似然率(negative log-likelihood, NLL)最小化。

这一策略基本可以应用在任何监督学习问题中:对于正确类型的输入/输出变量,写下其条件概率分布的参数族。

\subsection{支持向量机}
\label{sec:5.7.2}

支持向量机(Boser et al., 1992; Cortes and Vapnik, 1995)是最具影响力的监督学习方法之一。该方法与逻辑回归很相似,因为都是由线性函数$\omega^{T}x + b$所驱动。不同于逻辑回归,支持向量机(Support Vector Machine, SVM)并不提供概率值,只有分类结果。当$\omega^{T}x + b$为正,SVM预测为正类;同理当$\omega^{T}x + b$为负,则预测为负类。

支持向量机的关键创新点是\textbf{核技巧}(kernel trick)。核技巧观察到很多机器学习算法可以写作样本的点乘积。例如,支持向量机所用的线性函数可以写作形如
\begin{equation}
	\omega^{T}x + b = b + \sum_{i=1}^{m}{\alpha_{i}x^{T} x^{(i)}} 
    \label{form:5.81}
\end{equation}

这里$x^{(i)}$是一个训练样本,$\alpha$是系数矢量。

以这种方式重写学习算法之后,我们便可以用特征函数$\phi(x)$的输出和函数$k(\textbf{x}, \textbf{x}^{(i)})=\phi(\textbf{x})\cdot\phi(\textbf{x}^{(i)})$替代$\textbf{x}$,其中的$k(\textbf{x}, \textbf{x}^{(i)})=\phi(x)\cdot\phi(\textbf{x}^{(i)})$就叫做\textbf{核}(kernel)。$\cdot$操作符表示与$\phi(x)^{T}\phi(\textbf{x}^{(i)})$类似的内积。在有些特征空间中,我们可能无法使用真正的矢量内积;在有些无限多维的空间中,我们需要使用其他类型的内积,比如基于积分而不是加法的内积。此类内积的完整推导已经超出了本书的范畴。

用核替代了点积之后,我们可以用以下函数做预测
\begin{equation}
	f(x) = b + \sum{i}^{}{\alpha_{i}k(x,x^{(i}}
	\label{form:5.82}
\end{equation}

此函数对$textbf{x}$是非线性的,但是$\phi(\textbf{x})$和$f(\textbf{x}$之间是线性关系。并且$\alpha$和$f(\textbf{x}$也是线性关系。以下过程与基于核的方程都是严格等效的:对所有输入应用$\phi(\textbf{x}$,然后在新的变换空间中学习线性模型。

核技巧的强大有两重原因。首先,它允许我们并使用保证有效收敛的凸优化(convex optimization)技术,把对$x$的非线性函数当作线性的来学习。这是因为我们认为$\phi$是不变的,只优化$\alpha$,换言之,优化算法可以把决策方程在另一个空间中看作线性的。其次,相比于直接构建两个$\phi(\textbf{x})$矢量并显式求点积,核函数$k$的计算效率往往更高。

某些情况下,$\phi(\textbf{x})$甚至可以是无限维的,直接的显式求解将导致无穷的计算消耗。多数情况下,$k(\textbf{x}, \textbf{x'})$是$\textbf{x}$的非线性可解函数,即使$\phi(\textbf{x})$不可解。作为无限维特征空间中可解核的例子,我们构建一个特征映射,从非负整数x到$\phi(\textbf{x})$,设想该映射返回一个包含x个1及无穷多个0的矢量。我们可以写一个核函数$k(\textbf{x}, \textbf{x'}) = min(x, x^{i})$,这与无限维的点积严格等价。

最常用的核是高斯核(Gaussian kernel)
\begin{equation}
	k(u, v) = \mathbb{N}(u-v;0, \sigma^{2}I)
\end{equation}
$\mathbb{N}(x;\mu,\Sigma)$是标准正态密度。这个核也被称为径向基函数(radius basis function, RBF)核,因为其值在$v$空间中沿着$u$向外辐射而减小。高斯核对应着无限维空间里的点积,但是这一空间中的推导不像之前整数核的例子那样直观。

我们可以认为高斯核实现的是一种模板匹配(template matching)。一个与训练标签$y$相关的训练样本$x$构成了类$y$的一个模板。当测试点$x'$与$x$的欧几里得距离(Euclidean distance)很近的时候,高斯核有一个很大的响应,表示$x'$与$x$模板很相似。这一模型给相关训练标签$y$的权重很高。总体上看,预测是把基于对应模板样本(training example)进行过加权的训练标签(training label)组合了起来。

支持向量机并非唯一经由核技巧加强的算法,很多其他的线性模型都可以通过这种方式加强。这类搭载了核的算法也被称作核机器(kernel machine)或核方法(kernel method)。

核机器的最大缺在于,评估决策函数的计算量与训练样本数量呈线性关系,因为第$i$个样本向决策函数提供了$\alpha_{i}k(x,x^{(i)})$。支持向量机可以通过学习一个主要由0构成的矢量$alpha$来缓解这一弊端,对一个新样本做分类,只需要评估\alpha_{i}\neq0的样本,这些训练样本就是\textbf{支持向量}(support vector)。

核机器面临的另一大困难就是处理大数据时的超高计算资源消耗,我们将在\ref{sec:5.9}节重新审视该问题。普通的核机器很难提高适用性,我们将在\ref{sec:5.11}节重点讨论。现代深度学习的诞生正是为了突破这些局限,而当下的深度学习“复兴”正是始自Hinton et al(2006)展现了在MNIST数据集上,神经网络比RBF核支持向量机表现的更有力。

\subsection{其他简单的监督学习算法}
\label{sec:5.7.3}

我们已经简单了解过另一非概率的(non-probabilistic)监督学习算法,\textbf{近邻回归}(nearest neighbor regression)。更一般地来讲,k近邻(k-nearest neighbors)是一系列可用于分类和回归的技术。作为无参数学习算法,k近邻不为固定的参数量所限。我们通常认为k近邻算法没有任何参数,而是对训练数据施加了一个简单的函数。实际上k近邻甚至没有真正的训练或学习过程,在测试过程中,当我们想要对一个新测试输入$x$产生新输出$y$的时候,我们直接从训练数据$X$里找到离$x$最近的点,然后返回训练集对应$y$的平均值。在一个监督学习算法里,只要我们能定义出$y$的平均值,这个方法就是好用的。在分类问题中,我们可以对one-hot编码矢量$\textbf{c}$做平均,其中c_{y}=1且其他i值的c_{i}=0。


\section{非监督学习算法}
\label{sec:5.8}

\section{随机梯度下降法}
\label{sec:5.9}

\section{构建机器学习算法}
\label{sec:5.10}

\section{深度学习算法的动力}
\label{sec:5.11}
